% Options for packages loaded elsewhere
\PassOptionsToPackage{unicode}{hyperref}
\PassOptionsToPackage{hyphens}{url}
%
\documentclass[
]{article}
\usepackage{amsmath,amssymb}
\usepackage{lmodern}
\usepackage{iftex}
\ifPDFTeX
  \usepackage[T1]{fontenc}
  \usepackage[utf8]{inputenc}
  \usepackage{textcomp} % provide euro and other symbols
\else % if luatex or xetex
  \usepackage{unicode-math}
  \defaultfontfeatures{Scale=MatchLowercase}
  \defaultfontfeatures[\rmfamily]{Ligatures=TeX,Scale=1}
\fi
% Use upquote if available, for straight quotes in verbatim environments
\IfFileExists{upquote.sty}{\usepackage{upquote}}{}
\IfFileExists{microtype.sty}{% use microtype if available
  \usepackage[]{microtype}
  \UseMicrotypeSet[protrusion]{basicmath} % disable protrusion for tt fonts
}{}
\makeatletter
\@ifundefined{KOMAClassName}{% if non-KOMA class
  \IfFileExists{parskip.sty}{%
    \usepackage{parskip}
  }{% else
    \setlength{\parindent}{0pt}
    \setlength{\parskip}{6pt plus 2pt minus 1pt}}
}{% if KOMA class
  \KOMAoptions{parskip=half}}
\makeatother
\usepackage{xcolor}
\IfFileExists{xurl.sty}{\usepackage{xurl}}{} % add URL line breaks if available
\IfFileExists{bookmark.sty}{\usepackage{bookmark}}{\usepackage{hyperref}}
\hypersetup{
  pdftitle={Reproducible Research - Assignment 2},
  pdfauthor={Pete Olesen},
  hidelinks,
  pdfcreator={LaTeX via pandoc}}
\urlstyle{same} % disable monospaced font for URLs
\usepackage[margin=1in]{geometry}
\usepackage{color}
\usepackage{fancyvrb}
\newcommand{\VerbBar}{|}
\newcommand{\VERB}{\Verb[commandchars=\\\{\}]}
\DefineVerbatimEnvironment{Highlighting}{Verbatim}{commandchars=\\\{\}}
% Add ',fontsize=\small' for more characters per line
\usepackage{framed}
\definecolor{shadecolor}{RGB}{248,248,248}
\newenvironment{Shaded}{\begin{snugshade}}{\end{snugshade}}
\newcommand{\AlertTok}[1]{\textcolor[rgb]{0.94,0.16,0.16}{#1}}
\newcommand{\AnnotationTok}[1]{\textcolor[rgb]{0.56,0.35,0.01}{\textbf{\textit{#1}}}}
\newcommand{\AttributeTok}[1]{\textcolor[rgb]{0.77,0.63,0.00}{#1}}
\newcommand{\BaseNTok}[1]{\textcolor[rgb]{0.00,0.00,0.81}{#1}}
\newcommand{\BuiltInTok}[1]{#1}
\newcommand{\CharTok}[1]{\textcolor[rgb]{0.31,0.60,0.02}{#1}}
\newcommand{\CommentTok}[1]{\textcolor[rgb]{0.56,0.35,0.01}{\textit{#1}}}
\newcommand{\CommentVarTok}[1]{\textcolor[rgb]{0.56,0.35,0.01}{\textbf{\textit{#1}}}}
\newcommand{\ConstantTok}[1]{\textcolor[rgb]{0.00,0.00,0.00}{#1}}
\newcommand{\ControlFlowTok}[1]{\textcolor[rgb]{0.13,0.29,0.53}{\textbf{#1}}}
\newcommand{\DataTypeTok}[1]{\textcolor[rgb]{0.13,0.29,0.53}{#1}}
\newcommand{\DecValTok}[1]{\textcolor[rgb]{0.00,0.00,0.81}{#1}}
\newcommand{\DocumentationTok}[1]{\textcolor[rgb]{0.56,0.35,0.01}{\textbf{\textit{#1}}}}
\newcommand{\ErrorTok}[1]{\textcolor[rgb]{0.64,0.00,0.00}{\textbf{#1}}}
\newcommand{\ExtensionTok}[1]{#1}
\newcommand{\FloatTok}[1]{\textcolor[rgb]{0.00,0.00,0.81}{#1}}
\newcommand{\FunctionTok}[1]{\textcolor[rgb]{0.00,0.00,0.00}{#1}}
\newcommand{\ImportTok}[1]{#1}
\newcommand{\InformationTok}[1]{\textcolor[rgb]{0.56,0.35,0.01}{\textbf{\textit{#1}}}}
\newcommand{\KeywordTok}[1]{\textcolor[rgb]{0.13,0.29,0.53}{\textbf{#1}}}
\newcommand{\NormalTok}[1]{#1}
\newcommand{\OperatorTok}[1]{\textcolor[rgb]{0.81,0.36,0.00}{\textbf{#1}}}
\newcommand{\OtherTok}[1]{\textcolor[rgb]{0.56,0.35,0.01}{#1}}
\newcommand{\PreprocessorTok}[1]{\textcolor[rgb]{0.56,0.35,0.01}{\textit{#1}}}
\newcommand{\RegionMarkerTok}[1]{#1}
\newcommand{\SpecialCharTok}[1]{\textcolor[rgb]{0.00,0.00,0.00}{#1}}
\newcommand{\SpecialStringTok}[1]{\textcolor[rgb]{0.31,0.60,0.02}{#1}}
\newcommand{\StringTok}[1]{\textcolor[rgb]{0.31,0.60,0.02}{#1}}
\newcommand{\VariableTok}[1]{\textcolor[rgb]{0.00,0.00,0.00}{#1}}
\newcommand{\VerbatimStringTok}[1]{\textcolor[rgb]{0.31,0.60,0.02}{#1}}
\newcommand{\WarningTok}[1]{\textcolor[rgb]{0.56,0.35,0.01}{\textbf{\textit{#1}}}}
\usepackage{graphicx}
\makeatletter
\def\maxwidth{\ifdim\Gin@nat@width>\linewidth\linewidth\else\Gin@nat@width\fi}
\def\maxheight{\ifdim\Gin@nat@height>\textheight\textheight\else\Gin@nat@height\fi}
\makeatother
% Scale images if necessary, so that they will not overflow the page
% margins by default, and it is still possible to overwrite the defaults
% using explicit options in \includegraphics[width, height, ...]{}
\setkeys{Gin}{width=\maxwidth,height=\maxheight,keepaspectratio}
% Set default figure placement to htbp
\makeatletter
\def\fps@figure{htbp}
\makeatother
\setlength{\emergencystretch}{3em} % prevent overfull lines
\providecommand{\tightlist}{%
  \setlength{\itemsep}{0pt}\setlength{\parskip}{0pt}}
\setcounter{secnumdepth}{-\maxdimen} % remove section numbering
\ifLuaTeX
  \usepackage{selnolig}  % disable illegal ligatures
\fi

\title{Reproducible Research - Assignment 2}
\author{Pete Olesen}
\date{2022-05-17}

\begin{document}
\maketitle

\hypertarget{us-storm-events---largest-health-impact-and-financial-impact}{%
\subsection{US Storm Events - Largest health impact and financial
impact}\label{us-storm-events---largest-health-impact-and-financial-impact}}

\hypertarget{to-2011}{%
\subsection{1950 to 2011}\label{to-2011}}

\hypertarget{synopsis}{%
\subsection{Synopsis}\label{synopsis}}

Questions Your data analysis must address the following questions:

Across the United States, which types of events (as indicated in the
\color{red}{\verb|EVTYPE|}EVTYPE variable) are most harmful with respect
to population health?

Across the United States, which types of events have the greatest
economic consequences?

\begin{verbatim}
Consider writing your report as if it were to be read by a government or municipal manager who might be responsible for preparing for severe weather events and will need to prioritize resources for different types of events. However, there is no need to make any specific recommendations in your report.
\end{verbatim}

\hypertarget{data-processing}{%
\subsection{Data Processing}\label{data-processing}}

\begin{Shaded}
\begin{Highlighting}[]
\CommentTok{\# import libraries}

\FunctionTok{library}\NormalTok{(tidyverse)}
\end{Highlighting}
\end{Shaded}

\begin{verbatim}
## -- Attaching packages --------------------------------------- tidyverse 1.3.1 --
\end{verbatim}

\begin{verbatim}
## v ggplot2 3.3.6     v purrr   0.3.4
## v tibble  3.1.7     v dplyr   1.0.9
## v tidyr   1.2.0     v stringr 1.4.0
## v readr   2.1.2     v forcats 0.5.1
\end{verbatim}

\begin{verbatim}
## -- Conflicts ------------------------------------------ tidyverse_conflicts() --
## x dplyr::filter() masks stats::filter()
## x dplyr::lag()    masks stats::lag()
\end{verbatim}

\begin{Shaded}
\begin{Highlighting}[]
\FunctionTok{library}\NormalTok{(ggplot2)}
\FunctionTok{library}\NormalTok{(dplyr)}
\FunctionTok{library}\NormalTok{(readr)}
\FunctionTok{library}\NormalTok{(lubridate)}
\end{Highlighting}
\end{Shaded}

\begin{verbatim}
## 
## Attaching package: 'lubridate'
\end{verbatim}

\begin{verbatim}
## The following objects are masked from 'package:base':
## 
##     date, intersect, setdiff, union
\end{verbatim}

\begin{Shaded}
\begin{Highlighting}[]
\CommentTok{\# library(gtsummary)}
\CommentTok{\# library(janitor)}
\end{Highlighting}
\end{Shaded}

\begin{Shaded}
\begin{Highlighting}[]
\FunctionTok{getwd}\NormalTok{()}
\end{Highlighting}
\end{Shaded}

\begin{verbatim}
## [1] "C:/Users/Owner/Desktop/R Projects/Reproducible Research - Assignment 2"
\end{verbatim}

\begin{Shaded}
\begin{Highlighting}[]
\CommentTok{\# Load Csv and convert date to a date field}

\NormalTok{storm\_data\_df }\OtherTok{\textless{}{-}} \FunctionTok{read.csv}\NormalTok{(}\StringTok{"./repdata\_data\_StormData.csv.bz2"}\NormalTok{) }\SpecialCharTok{\%\textgreater{}\%} 
        \FunctionTok{mutate}\NormalTok{(}\AttributeTok{BGN\_DATE\_NEW =} \FunctionTok{mdy\_hms}\NormalTok{(BGN\_DATE), }\AttributeTok{YEAR =} \FunctionTok{year}\NormalTok{(BGN\_DATE\_NEW)) }

\FunctionTok{head}\NormalTok{(storm\_data\_df)}
\end{Highlighting}
\end{Shaded}

\begin{verbatim}
##   STATE__           BGN_DATE BGN_TIME TIME_ZONE COUNTY COUNTYNAME STATE  EVTYPE
## 1       1  4/18/1950 0:00:00     0130       CST     97     MOBILE    AL TORNADO
## 2       1  4/18/1950 0:00:00     0145       CST      3    BALDWIN    AL TORNADO
## 3       1  2/20/1951 0:00:00     1600       CST     57    FAYETTE    AL TORNADO
## 4       1   6/8/1951 0:00:00     0900       CST     89    MADISON    AL TORNADO
## 5       1 11/15/1951 0:00:00     1500       CST     43    CULLMAN    AL TORNADO
## 6       1 11/15/1951 0:00:00     2000       CST     77 LAUDERDALE    AL TORNADO
##   BGN_RANGE BGN_AZI BGN_LOCATI END_DATE END_TIME COUNTY_END COUNTYENDN
## 1         0                                               0         NA
## 2         0                                               0         NA
## 3         0                                               0         NA
## 4         0                                               0         NA
## 5         0                                               0         NA
## 6         0                                               0         NA
##   END_RANGE END_AZI END_LOCATI LENGTH WIDTH F MAG FATALITIES INJURIES PROPDMG
## 1         0                      14.0   100 3   0          0       15    25.0
## 2         0                       2.0   150 2   0          0        0     2.5
## 3         0                       0.1   123 2   0          0        2    25.0
## 4         0                       0.0   100 2   0          0        2     2.5
## 5         0                       0.0   150 2   0          0        2     2.5
## 6         0                       1.5   177 2   0          0        6     2.5
##   PROPDMGEXP CROPDMG CROPDMGEXP WFO STATEOFFIC ZONENAMES LATITUDE LONGITUDE
## 1          K       0                                         3040      8812
## 2          K       0                                         3042      8755
## 3          K       0                                         3340      8742
## 4          K       0                                         3458      8626
## 5          K       0                                         3412      8642
## 6          K       0                                         3450      8748
##   LATITUDE_E LONGITUDE_ REMARKS REFNUM BGN_DATE_NEW YEAR
## 1       3051       8806              1   1950-04-18 1950
## 2          0          0              2   1950-04-18 1950
## 3          0          0              3   1951-02-20 1951
## 4          0          0              4   1951-06-08 1951
## 5          0          0              5   1951-11-15 1951
## 6          0          0              6   1951-11-15 1951
\end{verbatim}

\begin{Shaded}
\begin{Highlighting}[]
\CommentTok{\# class(storm\_data\_df$BGN\_DATE\_NEW)}
\end{Highlighting}
\end{Shaded}

Let's take a look at the loaded data frame. I noticed that the EVTYPE -
Event type is a character field. I have read the documentation,
(10-1605\_StormDataPrep.pdf) and the field names in the data are not
specifically defined. I also read the FAQ (NCDC Storm Events FAQ.pdf)
and the assignment instructions. I don't have a high degree of
confidence in the data structure and what is contained in each field.

In any case, let's see how many unique EVTYPE events there are.

\begin{Shaded}
\begin{Highlighting}[]
\CommentTok{\# Let\textquotesingle{}s take a look at our data. I think the EVTYPE field has a far \# \# amount of cleaning to do.}
\CommentTok{\# How many unique values in EVTYPE}

\NormalTok{Unique\_EVtypes }\OtherTok{\textless{}{-}}\FunctionTok{unique}\NormalTok{(storm\_data\_df}\SpecialCharTok{$}\NormalTok{EVTYPE)}
\FunctionTok{length}\NormalTok{(Unique\_EVtypes)}
\end{Highlighting}
\end{Shaded}

\begin{verbatim}
## [1] 985
\end{verbatim}

Ouch! there are 985 unique event types. I looked through the list and it
looks like a lot of cleaning needs to be done here. For example, there
appears to be a lot of potentially redundant descriptions, ie
``Blizzard'' and ``Blizzard Weather''. This is but one example.

For the purposes of this assignment, I am not going to attempt to clean
this field. I would like to clean this data, but I fear that my
reclassification of EVTYPE would be arbitrary and not truly represent
the events as they occurred.

That being said, I will do the analysis with the EVTYPES as originally
presented.

\hypertarget{health-impact-by-event-type-evtype}{%
\subsection{Health impact by Event Type
(EVTYPE)}\label{health-impact-by-event-type-evtype}}

First to quantify the health impact. I am adding together the fatalities
and injuries as the measure for impact on the health of the population.

\begin{Shaded}
\begin{Highlighting}[]
\NormalTok{health\_impact\_df }\OtherTok{\textless{}{-}}\NormalTok{ storm\_data\_df }\SpecialCharTok{\%\textgreater{}\%}
      \FunctionTok{select}\NormalTok{ (YEAR, EVTYPE, FATALITIES, INJURIES) }\SpecialCharTok{\%\textgreater{}\%}
      \FunctionTok{group\_by}\NormalTok{(EVTYPE) }\SpecialCharTok{\%\textgreater{}\%} 
      \FunctionTok{summarise}\NormalTok{(}\AttributeTok{total\_health\_impact =} \FunctionTok{sum}\NormalTok{(INJURIES)}\SpecialCharTok{+}\FunctionTok{sum}\NormalTok{(FATALITIES))                  }\SpecialCharTok{\%\textgreater{}\%} 
                \FunctionTok{arrange}\NormalTok{(}\FunctionTok{desc}\NormalTok{(total\_health\_impact), }\AttributeTok{.by\_group =} \ConstantTok{TRUE}\NormalTok{)}

\FunctionTok{head}\NormalTok{(health\_impact\_df,}\DecValTok{10}\NormalTok{)}
\end{Highlighting}
\end{Shaded}

\begin{verbatim}
## # A tibble: 10 x 2
##    EVTYPE            total_health_impact
##    <chr>                           <dbl>
##  1 TORNADO                         96979
##  2 EXCESSIVE HEAT                   8428
##  3 TSTM WIND                        7461
##  4 FLOOD                            7259
##  5 LIGHTNING                        6046
##  6 HEAT                             3037
##  7 FLASH FLOOD                      2755
##  8 ICE STORM                        2064
##  9 THUNDERSTORM WIND                1621
## 10 WINTER STORM                     1527
\end{verbatim}

This, at first blush, doesn't feel right to me. Tornado's are 10X more
health impact than the other events. While tornado's can be very
destructive, they are also relatively rare. Need to look at the source
data\ldots{}

\begin{Shaded}
\begin{Highlighting}[]
\NormalTok{tornado\_df }\OtherTok{\textless{}{-}}   \FunctionTok{filter}\NormalTok{(storm\_data\_df, EVTYPE }\SpecialCharTok{==} \StringTok{"TORNADO"}\NormalTok{) }\SpecialCharTok{\%\textgreater{}\%}  
                \FunctionTok{select}\NormalTok{(BGN\_DATE\_NEW,EVTYPE, STATE, COUNTYNAME,INJURIES, FATALITIES)  }\SpecialCharTok{\%\textgreater{}\%}
                \FunctionTok{mutate}\NormalTok{(}\AttributeTok{tot\_health\_impact =}\NormalTok{ (INJURIES)}\SpecialCharTok{+}\NormalTok{ (FATALITIES))                 }\SpecialCharTok{\%\textgreater{}\%} 
                \FunctionTok{arrange}\NormalTok{(}\FunctionTok{desc}\NormalTok{(tot\_health\_impact))}

\FunctionTok{head}\NormalTok{(tornado\_df, }\DecValTok{10}\NormalTok{)}
\end{Highlighting}
\end{Shaded}

\begin{verbatim}
##    BGN_DATE_NEW  EVTYPE STATE COUNTYNAME INJURIES FATALITIES tot_health_impact
## 1    1979-04-10 TORNADO    TX    WICHITA     1700         42              1742
## 2    1953-06-09 TORNADO    MA  WORCESTER     1228         90              1318
## 3    2011-05-22 TORNADO    MO     JASPER     1150        158              1308
## 4    1974-04-03 TORNADO    OH     GREENE     1150         36              1186
## 5    1953-06-08 TORNADO    MI    GENESEE      785        116               901
## 6    2011-04-27 TORNADO    AL TUSCALOOSA      800         44               844
## 7    2011-04-27 TORNADO    AL  JEFFERSON      700         20               720
## 8    1953-05-11 TORNADO    TX   MCLENNAN      597        114               711
## 9    1965-04-11 TORNADO    IN     HOWARD      560         17               577
## 10   1966-03-03 TORNADO    MS      HINDS      504         57               561
\end{verbatim}

As it turns out, this may well be reasonable data. As you can see from
the tornado\_df, the largest health impact was from the Witchita, TX
tornado in April 1979. I did a Google search, and in fact there were 42
fatalities from this event.

Here is the Wikipedia link: {[}Witchita County, TX
tornado{]}(\url{https://en.wikipedia.org/wiki/1979_Red_River_Valley_tornado_outbreak\#Aftermath}

So my previous assumption was not correct. For this assignment, I am
going to rely on the data as it stands.

\hypertarget{financial-impact-by-event-type-evtype}{%
\subsection{Financial impact by Event Type
(EVTYPE)}\label{financial-impact-by-event-type-evtype}}

Looking at the data, there are two types of damage quantified. First is
the amount of property damage PROPDMG and crop damage CROPDMG. The
second is the rate of property damage or crop damage (ie K = thousands,
M = millions, B = Billions). Each of these is stored in a separate
field.

\begin{Shaded}
\begin{Highlighting}[]
\CommentTok{\#First, create a df to pull the necessary data. Create the }


\NormalTok{financial\_impact\_df }\OtherTok{\textless{}{-}} \FunctionTok{filter}\NormalTok{(storm\_data\_df,PROPDMG }\SpecialCharTok{\textgreater{}}\DecValTok{0} \SpecialCharTok{|}\NormalTok{CROPDMG }\SpecialCharTok{\textgreater{}} \DecValTok{0}\NormalTok{) }\SpecialCharTok{\%\textgreater{}\%}
      \FunctionTok{select}\NormalTok{ (YEAR, EVTYPE, PROPDMG, PROPDMGEXP, CROPDMG, CROPDMGEXP) }\SpecialCharTok{\%\textgreater{}\%}
        \FunctionTok{mutate}\NormalTok{(}\AttributeTok{PROPDMG\_RATE =}   \FunctionTok{gsub}\NormalTok{(}\StringTok{\textquotesingle{}K\textquotesingle{}}\NormalTok{,}\DecValTok{1000}\NormalTok{,}
                                \FunctionTok{gsub}\NormalTok{(}\StringTok{\textquotesingle{}M\textquotesingle{}}\NormalTok{,}\DecValTok{1000000}\NormalTok{,}
                                \FunctionTok{gsub}\NormalTok{(}\StringTok{\textquotesingle{}B\textquotesingle{}}\NormalTok{,}\DecValTok{1000000000}\NormalTok{,PROPDMGEXP))),}
               \AttributeTok{CROPDMG\_RATE =}   \FunctionTok{gsub}\NormalTok{(}\StringTok{\textquotesingle{}K\textquotesingle{}}\NormalTok{,}\DecValTok{1000}\NormalTok{,}
                                \FunctionTok{gsub}\NormalTok{(}\StringTok{\textquotesingle{}M\textquotesingle{}}\NormalTok{,}\DecValTok{1000000}\NormalTok{,}
                                \FunctionTok{gsub}\NormalTok{(}\StringTok{\textquotesingle{}B\textquotesingle{}}\NormalTok{,}\DecValTok{1000000000}\NormalTok{,CROPDMGEXP))),}
               \AttributeTok{PROPDMG\_RATE =} \FunctionTok{as.numeric}\NormalTok{(PROPDMG\_RATE),}
               \AttributeTok{CROPDMG\_RATE =} \FunctionTok{as.numeric}\NormalTok{(CROPDMG\_RATE),}
               \AttributeTok{PROPDMG\_RATE =} \FunctionTok{coalesce}\NormalTok{(PROPDMG\_RATE,}\DecValTok{0}\NormalTok{),}
               \AttributeTok{CROPDMG\_RATE =} \FunctionTok{coalesce}\NormalTok{(CROPDMG\_RATE,}\DecValTok{0}\NormalTok{),}
               \AttributeTok{TOT\_PROP\_DMG\_USD =}\NormalTok{ PROPDMG\_RATE}\SpecialCharTok{*}\NormalTok{PROPDMG,}
               \AttributeTok{TOT\_CROP\_DMG\_USD =}\NormalTok{ CROPDMG\_RATE}\SpecialCharTok{*}\NormalTok{CROPDMG,}
               \AttributeTok{TOT\_DMG\_USD =}\NormalTok{ TOT\_PROP\_DMG\_USD }\SpecialCharTok{+}\NormalTok{ TOT\_CROP\_DMG\_USD}
\NormalTok{               )}
\end{Highlighting}
\end{Shaded}

\begin{verbatim}
## Warning in mask$eval_all_mutate(quo): NAs introduced by coercion

## Warning in mask$eval_all_mutate(quo): NAs introduced by coercion
\end{verbatim}

\begin{Shaded}
\begin{Highlighting}[]
\FunctionTok{glimpse}\NormalTok{(financial\_impact\_df)                                       }
\end{Highlighting}
\end{Shaded}

\begin{verbatim}
## Rows: 245,031
## Columns: 11
## $ YEAR             <dbl> 1950, 1950, 1951, 1951, 1951, 1951, 1951, 1952, 1952,~
## $ EVTYPE           <chr> "TORNADO", "TORNADO", "TORNADO", "TORNADO", "TORNADO"~
## $ PROPDMG          <dbl> 25.0, 2.5, 25.0, 2.5, 2.5, 2.5, 2.5, 2.5, 25.0, 25.0,~
## $ PROPDMGEXP       <chr> "K", "K", "K", "K", "K", "K", "K", "K", "K", "K", "M"~
## $ CROPDMG          <dbl> 0, 0, 0, 0, 0, 0, 0, 0, 0, 0, 0, 0, 0, 0, 0, 0, 0, 0,~
## $ CROPDMGEXP       <chr> "", "", "", "", "", "", "", "", "", "", "", "", "", "~
## $ PROPDMG_RATE     <dbl> 1e+03, 1e+03, 1e+03, 1e+03, 1e+03, 1e+03, 1e+03, 1e+0~
## $ CROPDMG_RATE     <dbl> 0, 0, 0, 0, 0, 0, 0, 0, 0, 0, 0, 0, 0, 0, 0, 0, 0, 0,~
## $ TOT_PROP_DMG_USD <dbl> 25000, 2500, 25000, 2500, 2500, 2500, 2500, 2500, 250~
## $ TOT_CROP_DMG_USD <dbl> 0, 0, 0, 0, 0, 0, 0, 0, 0, 0, 0, 0, 0, 0, 0, 0, 0, 0,~
## $ TOT_DMG_USD      <dbl> 25000, 2500, 25000, 2500, 2500, 2500, 2500, 2500, 250~
\end{verbatim}

\begin{Shaded}
\begin{Highlighting}[]
      \CommentTok{\# group\_by(EVTYPE)}
\end{Highlighting}
\end{Shaded}

\hypertarget{results}{%
\subsection{Results}\label{results}}

\end{document}
